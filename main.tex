\documentclass[usenames,dvipsnames]{beamer}
\addtocontents{toc}{\protect\setcounter{tocdepth}{2}}
\usetheme{Berlin}
\usepackage[utf8]{inputenc}
\usepackage[english]{babel}
\usepackage{graphicx}
\usepackage{url}
\usepackage{fancyvrb}
\usepackage{multicol}
\usepackage{relsize}

\graphicspath{ {images/} }

\title{Reducing the Binary Image Footprint}
\subtitle{Techniques to Produce Smaller Binaries}
\author{Alexander Livenets}
\institute{}
\date{29 May 2019}

\newcommand{\boldgreen}[1] {\textbf{\color{Green}{#1}}}
\newcommand{\codeinline}[1] {\texttt{\smaller[2]{#1}}}

\begin{document}

\AtBeginSection[]
{
\begin{frame}
\frametitle{Table of Contents}
\tableofcontents[currentsection]
\end{frame}
}

\begin{frame}
\titlepage
\end{frame}

\section{Introduction}

\subsection{Motivation}
\begin{frame}
\frametitle{\subsecname}
\begin{itemize}
	\item Faster boot-up
	\item Faster software update
	\item More space for user data and Rich OS
	\item Better performance (in some cases)
	\item Bootloaders, bare metal, ECU firmware development
\end{itemize}
\note{There is always a lot of attention nowadays paid to how to optimize performance and RAM usage, but only a little to how to minimize your executable footprint. Indeed, now all resources like Flash, HDDs, SDDs are considered almost unlimited. But in embedded, people still care much about Flash. So, here are the reasons why we need it.}
\end{frame}

\subsection{Drawbacks}
\begin{frame}
\frametitle{\subsecname}
\begin{itemize}
	\item Increased compilation time
	\item Some corner cases where optimization may fail
\end{itemize}
\end{frame}

\section{Strategies}
\subsection{Yocto Level}
\begin{frame}
\frametitle{System minification}
\begin{itemize}
	\item Yocto: Do not install documentation, development headers, etc. \pause
	\item Locales and fonts: Cut them out, you ain't gonna need much in console, live with C UTF8 locale and one single font \pause
	\item Use hardware-specific optimizations
	\begin{itemize} 
		\item e.g. ARM:hard FP instead of soft FP \pause
	\end{itemize}
	\item Libc: Use smaller libc alternatives (bionic, musl, uclibc) \pause
	\item g++: Use newer compiler version \pause
	\item Clang: Compile with clang and libc++ for middleware and application services
\end{itemize}
\end{frame}

\begin{frame}
\frametitle{System minification}
\begin{itemize}
	\item Linux: Strip libraries and executables \pause
	\item CMake: Build minimal release, \codeinline{CMAKE\_BUILD\_TYPE=MinSizeRel} \pause
	\item Kernel: Cut out unused modules (network, wifi, etc.) \pause
	\item Kernel: Build tiny kernel: \url{https://tiny.wiki.kernel.org/} \pause
	\item System: Analyze used libraries and components, cut out unused libraries (e.g. some of Qt), use single library for one function Examples (e.g. JSON, XML parsing, HTTP, etc.) 
\end{itemize}
\end{frame}

\begin{frame}
\frametitle{Data minification}
\begin{itemize}
	\item Minify/compress scripts and markup/configuration files (JSON, XML, HTML, QML, CSS, JavaScript) \pause
	\item Images: Use image lossless compression  \pause
	\item Compile QML sources into one binary/library 
\end{itemize}
\end{frame}

\subsection{Service Level}

\subsubsection{Compilation and linking}

\begin{frame}
\frametitle{\subsubsecname}
\centering
\includegraphics[scale=0.35]{CompilationDiagram.png}
\end{frame}

\begin{frame}
\frametitle{\subsubsecname}
\begin{itemize}
	\item \codeinline{.text} - compiled code (read-only)
	\item \codeinline{.rodata} - variable initialization data + constants + strings  -\textgreater \codeinline{.text} (read-only)
	\item \codeinline{.data} - all global and static variables (read-write)
	\item \codeinline{.bss} - "Better Save Space", zero-initialized variables
\end{itemize}

\centering
\includegraphics[scale=0.35]{RelocatedObject.png}
\end{frame}

\begin{frame}[fragile]
\frametitle{Compilation and linking}
\begin{verbatim}
$ size AudioManager
text	   data	    bss	    dec	    hex	filename
650175	  12808	   3912	 666895	  a2d0f	AudioManager

$ du -b AudioManager
969632	AudioManager
\end{verbatim}
\end{frame}

\subsubsection{Compile-time feature toggle}

\begin{frame}[fragile]
\frametitle{\subsubsecname}
\small
\begin{multicols}{2}
main.cpp
\begin{verbatim}
#ifdef FEATURE
class FeatureImpl {
//...	
};

#endif

int main(void) {
#ifdef FEATURE
registerFeature(new FeatureImpl());
#endif
//other code...
}
\end{verbatim}

\columnbreak

CMakeLists.txt:
\begin{verbatim}
option(FEATURE "Some feature" ON)
$ cmake .. -DFEATURE=OFF
\end{verbatim}
\end{multicols}
\end{frame}

\subsubsection{Compilation Options}

\begin{frame}
\frametitle{\subsubsecname}
\begin{itemize}
	\item \codeinline{-Os} - smaller and (maybe) faster than -O3 \pause
	\item \codeinline{-s} - strip resulting binary object in gcc \pause
	\item \codeinline{-mtune} - use CPU-specific compiler optimizations. Enables usage of extended CPU instruction set \pause
	\item \codeinline{-fno-unroll-loops} - do not unroll loops, may produce less code. \pause
	\item \codeinline{-fno-inline-small-functions -finline-functions-called-once} - optimization of inline functions, inline in certain cases \pause
	\item \codeinline{-fshort-enums} – shrink enums size to max contained value; WARNING: may break ABI (USE ONLY for static binaries) 
\end{itemize}
\end{frame}

\subsubsection{Link Options}

\begin{frame}
\frametitle{\subsubsecname}
\begin{itemize}
	\item \codeinline{-ffunction-sections -fdata-sections -Wl,--gc-sections} - every function and variable goes to its own section, then unused sections are thrown out \pause
	\item \codeinline{strip --strip-all} on the final executable \pause
	\item \codeinline{-flto} - LTO, link-time optimization, allows to optimize code across multiple object files \pause
	\item \codeinline{strip --remove-section=.comment --remove-section=.note} - removing many copies of unneeded strings like "GCC 4.0.1 20050727 (Ubuntu 18.04)" 
\end{itemize}
\end{frame}

\subsubsection{Code Refactoring}

\begin{frame}
\frametitle{\subsubsecname}
\begin{itemize}
	\item Static vs. dynamic linking
	\begin{itemize} 
		\item Static libraries: better, if used by one application
		\begin{itemize}
			\item Split functions to source files, the resulting binary will be smaller and better optimized (see libc.a, libm.a). \\ WARNING: use static linking with caution
		\end{itemize}
		\item Dynamic libraries: less code size if used by multiple applications
		\item Link only used libraries
	\end{itemize}
\end{itemize}
\end{frame}

\begin{frame}[fragile]
\footnotesize
\begin{itemize}
	\item Reasonable logging. Logs often use similar strings:
	\begin{Verbatim}[commandchars=\\\{\}]
	$ strings AudioManager
	...
	DBusWrapper::DBusWrapper DBus Connection is \textcolor{red}{null}
	DBusWrapper::DBusWrapper DBus Connection is
	...
	DBusWrapper::DBusWrapper Registering of \textcolor{red}{watch} functions failed
	DBusWrapper::DBusWrapper Registering of \textcolor{red}{timer} functions failed
	\end{Verbatim}
\end{itemize}
\end{frame}

\begin{frame}[fragile]
\frametitle{\subsubsecname}
\begin{itemize}
	\item \codeinline{struct}/\codeinline{class} alignment and packing \pause
	\item Use less C++ templates \pause
	\item Pay attention to \codeinline{inline} functions \pause
	\item Do not use C++ exceptions, compiler option: \codeinline{-fno-exceptions} \pause
	\item Do not use C++ RTTI, compiler option: \codeinline{-fno-rtti} \pause
	\item Statically initialized large arrays and C++ objects with custom (nonzero) data. The initialization data usually goes to \codeinline{.text} block. 
	\small
	\begin{verbatim}
	static const std::vector<T> -> static const T[]
	static const std::string -> static const char*
	\end{verbatim}
	\normalsize
\end{itemize}
\end{frame}

\subsubsection{Compression}

\begin{frame}
\frametitle{\subsubsecname}
\center
\Large UPX \normalsize \\
\url{https://upx.github.io} \\
In-place decompression of compressed executable/library
\end{frame}

\subsubsection{Disable Optimization}

\begin{frame}
\frametitle{\subsubsecname}
\begin{itemize}
	\item \codeinline{volatile} keyword
	\item \codeinline{\#pragma GCC optimize("O0")}
\end{itemize}
\end{frame}

\section{Real-life Example}
\begin{frame}
\frametitle{\secname}
\center
\Large GENIVI AudioManager \normalsize \\
\url{https://github.com/GENIVI/AudioManager.git} \\
Intel Core i7, x86-64, Linux, gcc 7.4.0, clang 6.0.0
\end{frame}

\begin{frame}
\frametitle{\secname}
\small
GCC:
\begin{itemize} 
	\item \codeinline{cmake .. -DWITH\_CAPI\_WRAPPER=OFF -DWITH\_DBUS\_WRAPPER=ON} 
	\item \footnotesize{Flags:} \codeinline{-fno-unroll-loops -fno-inline-small-functions \\-finline-functions-called-once}
\end{itemize}
Clang: 
\begin{itemize}
	\item \codeinline{CC=/usr/bin/clang CXX=/usr/bin/clang++ cmake .. -DCMAKE\_USER\_MAKE\_RULES\_OVERRIDE=\$(pwd)/../ClangDefinitions.txt -DWITH\_CAPI\_WRAPPER=OFF -DWITH\_DBUS\_WRAPPER=ON}
	\item \footnotesize{Flags:} \codeinline{-fvectorize}
\end{itemize}

Static linking with Core and Utilities library
\end{frame}

\begin{frame}[fragile]
\frametitle{\secname}
\begin{table}
\resizebox{1.05\textwidth}{!}{%
\begin{tabular}{|c|c|c|c|c|c|l|c|l|c|}
\hline
Release & MinSizeRel & -march=corei7 & Section GC flags & Misc flags & LTO & \multicolumn{1}{|c|}{Clang} & Win  & \multicolumn{1}{|c|}{gcc}  & Win \\ \hline
\checkmark &  & &  & & & \begin{tabular}[l]{@{}l@{}}1020K (not stripped)\\ 780K (stripped)\end{tabular} & & \begin{tabular}[l]{@{}l@{}}1.1M (not stripped)\\ 819K (stripped)\end{tabular} & \\ \hline
& \checkmark & & & & & \begin{tabular}[l]{@{}l@{}}\textbf{\color{red}{1.1M (not stripped)}} \\ 764K (stripped)\end{tabular} & \boldgreen{-2.05\%} & \begin{tabular}[l]{@{}l@{}}947K (not stripped)\\ 651K (stripped)\end{tabular} & \boldgreen{-20.5\%} \\ \hline
& \checkmark & \checkmark & & & & 764K (stripped) & \boldgreen{-2.05\%} & 651K (stripped) & \boldgreen{-20.5\%} \\ \hline
& \checkmark & \checkmark & \checkmark & & & 764K (stripped) & \boldgreen{-2.05\%} & 651K (stripped) & \boldgreen{-20.5\%} \\ \hline
& \checkmark & \checkmark & \checkmark & \checkmark & & 764K (stripped) & \boldgreen{-2.05\%} & 651K (stripped) & \boldgreen{-20.5\%} \\ \hline
& \checkmark & & & & \checkmark & \textbf{\color{red}{FAILED}} & & 500K (stripped) & \boldgreen{-38.9\%} \\ \hline
& \checkmark & \checkmark & \checkmark & \checkmark & \checkmark & \textbf{\color{red}{FAILED}} & & 499K (stripped) & \boldgreen{-39\%} \\ \hline
\end{tabular}%
}
\end{table}
\footnotesize

Release: \texttt{-O2}

MinSizeRel: \texttt{-Os -s}

Section GC flags: \texttt{-ffunction-sections -fdata-sections -Wl,--gc-sections}

Misc flags: \texttt{-fno-unroll-loops -fno-inline-small-functions -finline-functions-called-once}
\end{frame}

\section{Further Research}

\begin{frame}
\frametitle{\secname}
\begin{itemize}
	\item Yocto tinification on real example
	\item Test on Aarch64 architecture
	\item Run optimization for current Yocto image
	\item Clang LTO + advanced optimization research
\end{itemize}
\end{frame}

\section{Resources}

\begin{frame}
\frametitle{\secname}
\tiny

Executable size reduction techniques in Linux

\begin{itemize}
	\item \url{https://wiki.wxwidgets.org/Reducing_Executable_Size}
	\item \url{http://www.catch22.net/tuts/win32/reducing-executable-size}
	\item \url{http://ptspts.blogspot.com/2013/12/how-to-make-smaller-c-and-c-binaries.html}
	\item \url{https://elinux.org/images/0/07/Opdenacker-embedded-linux-size-reduction-techniques.pdf}
	\item \url{https://s3-eu-west-1.amazonaws.com/downloads-mips/mips-documentation/login-required/using_gcc_toolchain_options_to_optimize_code_size.pdf}
	\item \url{https://medium.com/@aka.rider/how-to-optimize-c-and-c-code-in-2018-bd4f90a72c2b}
	\item \url{http://www.muppetlabs.com/~breadbox/software/tiny/teensy.html}
	\item \url{http://events17.linuxfoundation.org/sites/events/files/slides/elc14_raj.pdf}
	\item \url{https://elinux.org/System_Size\#Hand-optimizing_programs.2C_for_size}
	\item \url{https://blog.qt.io/blog/2019/01/02/qt-applications-lto/}
\end{itemize}
\end{frame}

\begin{frame}
\frametitle{\secname}
\tiny

Libc comparison

\begin{itemize}
	\item \url{http://www.etalabs.net/compare_libcs.html}
	\item \url{https://events.static.linuxfound.org/sites/events/files/slides/libc-talk.pdf}
\end{itemize}

GCC optimization options

\begin{itemize}
	\item \url{http://gcc.gnu.org/onlinedocs/gcc/Optimize-Options.html}
	\item \url{https://gcc.gnu.org/onlinedocs/gcc/Code-Gen-Options.html}
	\item \url{https://gcc.gnu.org/onlinedocs/gcc/C_002b_002b-Dialect-Options.html}
\end{itemize}

Shrinking the linux kernel

\begin{itemize}
	\item \url{https://lwn.net/Articles/744507/} - "Shrinking the kernel with link-time optimization"
	\item \url{https://lwn.net/Articles/746780/} - "Shrinking the kernel with an axe"
\end{itemize}

Yocto tinification

\begin{itemize}
	\item \url{https://elinux.org/images/2/2b/Elce11_hart.pdf}
	\item \url{https://elinux.org/images/5/54/Tom.zanussi-elc2014.pdf}
	\item \url{https://wiki.yoctoproject.org/wiki/Poky-Tiny}
\end{itemize}
\end{frame}

\begin{frame}
\frametitle{Links}
\footnotesize
GENIVI AudioManager fork: \\ \url{https://github.com/alivenets/AudioManager.git}
\newline \newline
Presentation sources: \\ \url{https://github.com/alivenets/binary-footprint-optimize-latex}
\end{frame}

\begin{frame}{}
\center \Huge Thanks for your attention!
\end{frame}

\end{document}
